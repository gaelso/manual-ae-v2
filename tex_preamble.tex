
%% Set font sf
%\renewcommand{\familydefault}{\sfdefault}

%% Original preamble from Yihui
\usepackage{booktabs}
\usepackage{float}
\usepackage{makecell}
\makeatletter
\def\thm@space@setup{%
  \thm@preskip=8pt plus 2pt minus 4pt
  \thm@postskip=\thm@preskip
}
\makeatother

%% This is for the watermark
%%\usepackage{draftwatermark}

%% This is for landscape tables
\usepackage{longtable}
\usepackage{pdflscape}

%% This is for multirow tables
\usepackage{multirow}


%% -------
%% Commands kept from the Book Original layout TeX file
%% ------

%% Setup from the original TeX layout preserved
\usepackage{pdfpages}
\usepackage[grey,utopia]{quotchap}
\usepackage{marvosym}

%% page ref for citations
%\usepackage[colorlinks=true,linkcolor=blue,citecolor=blue,pagebackref]{hyperref}
%\usepackage[pagebackref]{hyperref}
%\usepackage[backref=page]{hyperref}
%\usepackage[pageref]{backref}
%\usepackage{url}


%% Style of header and footers
\usepackage{fancyhdr}
\fancyhead{}%
\fancyhead[LE]{\thepage}%
\fancyhead[RE]{\leftmark}%
\fancyhead[LO]{\rightmark}%
\fancyhead[RO]{\thepage}%
\fancyfoot{}%
\pagestyle{fancy}

%% Remove headers when empty page before new chapter
\usepackage{emptypage}


\makeatletter
%% Redefinit le style des en-tetes en petites capitales
\renewcommand{\chaptermark}[1]{
        \markboth{\sc\chaptername\ \thechapter.\ #1}{}}%% avec fancyhdr
\renewcommand{\sectionmark}[1]{
        \markright{\sc\thesection\ #1}}%% avec fancyhdr

%% Pour mettre toutes les titres de section en police sans serif (modification de la classe 'book')
\renewcommand\section{\@startsection {section}{1}{\z@}%
                                   {-3.5ex \@plus -1ex \@minus -.2ex}%
                                   {2.3ex \@plus.2ex}%
                                   {\sffamily\Large\bfseries}}
\renewcommand\subsection{\@startsection{subsection}{2}{\z@}%
                                   {-3.25ex\@plus -1ex \@minus -.2ex}%
                                   {1.5ex \@plus .2ex}%
                                   {\sffamily\large\bfseries}}
\renewcommand\subsubsection{\@startsection{subsubsection}{3}{\z@}%
                                   {-3.25ex\@plus -1ex \@minus -.2ex}%
                                   {1.5ex \@plus .2ex}%
                                   {\sffamily\normalsize\bfseries}}
\renewcommand\paragraph{\@startsection{paragraph}{4}{\z@}%
                                   {3.25ex \@plus1ex \@minus.2ex}%
                                   {-1em}%
                                   {\sffamily\normalsize\bfseries}}
\renewcommand\subparagraph{\@startsection{subparagraph}{5}{\parindent}%
                                   {3.25ex \@plus1ex \@minus .2ex}%
                                   {-1em}%
                                   {\sffamily\normalsize\bfseries}}

%% (modification de la classe 'book.cls') -- Add content as bookmark to pdf
\def\chapterannex#1{\chapter*{#1
%        \@mkboth{\sc\small #1}{\sc\small #1}%% sans fancyhdr
        \@mkboth{\sc #1}{\sc #1}%% avec fancyhdr
        \addcontentsline{toc}{chapter}{\numberline {}#1}}%% ligne modifi?e
    \thispagestyle{plain}}%% ligne ajout?e
% \renewcommand\chapter*{
% %        \@mkboth{\sc\small #1}{\sc\small #1}%% sans fancyhdr
%         \@mkboth{}{}%% avec fancyhdr
%         %\addcontentsline{toc}{chapter}{\numberline {}#1}
%         }%% ligne modifi?e
%     %\thispagestyle{plain}}%% ligne ajout?e
\renewcommand\tableofcontents{%
    \if@twocolumn\@restonecoltrue\onecolumn\else\@restonecolfalse\fi
    \chapterannex{\contentsname}
    \@starttoc{toc}%
    \if@restonecol\twocolumn\fi}
\renewcommand\listoffigures{%
    \if@twocolumn\@restonecoltrue\onecolumn\else\@restonecolfalse\fi
    \chapterannex{\listfigurename}
    \@starttoc{lof}%
    \if@restonecol\twocolumn\fi}
\renewcommand\listoftables{%
    \if@twocolumn\@restonecoltrue\onecolumn\else\@restonecolfalse\fi
    \chapterannex{\listtablename}
    \@starttoc{lot}%
    \if@restonecol\twocolumn\fi}

\makeatother

%% Define glos for the glossary items
\newcommand{\gloss}[2]{\paragraph*{#1.} #2}

%% ------
%% END of book original layout.tex
%% ------

                     
%% Stop make title from placing YAML title before the cover page
\let\oldmaketitle\maketitle
\AtBeginDocument{\let\maketitle\relax}

%% Box 
\usepackage[breakable, skins]{tcolorbox}
\definecolor{boxcolor}{RGB}{255,255,255}
\newtcolorbox{filrouge}{
  breakable,
  enhanced,
  colback=boxcolor,
  colframe=red,
  %coltext=white,
  boxsep=5pt,
  arc=4pt}
  

%% sub figures
\usepackage{subfig}

%% Italic caption text
\usepackage[labelsep=endash,textfont=sl]{caption}

%% change verbatim output for R chunks
\usepackage{fancyvrb}
\usepackage[dvipsnames]{xcolor}
\usepackage{fvextra} %% Required fro breakline and breakanywhere
\RecustomVerbatimEnvironment{verbatim}{Verbatim}{commandchars=\\\{\}, fontsize=\small, formatcom=\color{blue}, breaklines=true, breakanywhere=true}


%% Modify the r chunk style in pdf/Tex to be all red
\usepackage{framed}
\DefineVerbatimEnvironment{Highlighting}{Verbatim}{commandchars=\\\{\}, fontsize=\small}
\definecolor{shadecolor}{RGB}{240,240,240} %247 originally
\makeatletter
\@ifundefined{Shaded}{
}{\renewenvironment{Shaded}{\begin{snugshade}}{\end{snugshade}}
\renewcommand{\AlertTok}[1]{\textcolor{red}{\textbf{#1}}}
\renewcommand{\AnnotationTok}[1]{\textcolor{red}{\textbf{\textit{#1}}}}
\renewcommand{\AttributeTok}[1]{\textcolor{red}{#1}}
\renewcommand{\BaseNTok}[1]{\textcolor{red}{#1}}
\renewcommand{\BuiltInTok}[1]{#1}
\renewcommand{\CharTok}[1]{\textcolor{red}{#1}}
\renewcommand{\CommentTok}[1]{\textcolor{red}{\textit{#1}}}
\renewcommand{\CommentVarTok}[1]{\textcolor{red}{\textbf{\textit{#1}}}}
\renewcommand{\ConstantTok}[1]{\textcolor{red}{#1}}
\renewcommand{\ControlFlowTok}[1]{\textcolor{red}{\textbf{#1}}}
\renewcommand{\DataTypeTok}[1]{\textcolor{red}{#1}}
\renewcommand{\DecValTok}[1]{\textcolor{red}{#1}}
\renewcommand{\DocumentationTok}[1]{\textcolor{red}{\textit{#1}}}
\renewcommand{\ErrorTok}[1]{\textcolor{red}{\textbf{#1}}}
\renewcommand{\ExtensionTok}[1]{#1}
\renewcommand{\FloatTok}[1]{\textcolor{red}{#1}}
\renewcommand{\FunctionTok}[1]{\textcolor{red}{#1}}
\renewcommand{\ImportTok}[1]{#1}
\renewcommand{\InformationTok}[1]{\textcolor{red}{\textbf{\textit{#1}}}}
\renewcommand{\KeywordTok}[1]{\textcolor{red}{\textbf{#1}}}
\renewcommand{\NormalTok}[1]{\textcolor{red}{#1}}
\renewcommand{\OperatorTok}[1]{\textcolor{red}{\textbf{#1}}}
\renewcommand{\OtherTok}[1]{\textcolor{red}{\textbf{#1}}}
\renewcommand{\PreprocessorTok}[1]{\textcolor{red}{#1}}
\renewcommand{\RegionMarkerTok}[1]{#1}
\renewcommand{\SpecialCharTok}[1]{\textcolor{red}{#1}}
\renewcommand{\SpecialStringTok}[1]{\textcolor{red}{#1}}
\renewcommand{\StringTok}[1]{\textcolor{red}{#1}}
\renewcommand{\VariableTok}[1]{\textcolor{red}{#1}}
\renewcommand{\VerbatimStringTok}[1]{\textcolor{red}{#1}}
\renewcommand{\WarningTok}[1]{\textcolor{red}{\textbf{\textit{#1}}}}
}
\makeatother


% %% Modify the r chunk style in pdf/Tex to mimic pygment in html + added automatic breaks when R outputs are longer than the page size
% \usepackage{framed}
% \DefineVerbatimEnvironment{Highlighting}{Verbatim}{commandchars=\\\{\}, fontsize=\small}
% \definecolor{shadecolor}{RGB}{240,240,240} %247 originally
% \makeatletter
% \@ifundefined{Shaded}{
% }{\renewenvironment{Shaded}{\begin{snugshade}}{\end{snugshade}}
% \renewcommand{\AlertTok}[1]{\textcolor[rgb]{1.00,0.00,0.00}{\textbf{#1}}}
% \renewcommand{\AnnotationTok}[1]{\textcolor[rgb]{0.38,0.63,0.69}{\textbf{\textit{#1}}}}
% \renewcommand{\AttributeTok}[1]{\textcolor[rgb]{0.49,0.56,0.16}{#1}}
% \renewcommand{\BaseNTok}[1]{\textcolor[rgb]{0.25,0.63,0.44}{#1}}
% \renewcommand{\BuiltInTok}[1]{#1}
% \renewcommand{\CharTok}[1]{\textcolor[rgb]{0.25,0.44,0.63}{#1}}
% \renewcommand{\CommentTok}[1]{\textcolor[rgb]{0.38,0.63,0.69}{\textit{#1}}}
% \renewcommand{\CommentVarTok}[1]{\textcolor[rgb]{0.38,0.63,0.69}{\textbf{\textit{#1}}}}
% \renewcommand{\ConstantTok}[1]{\textcolor[rgb]{0.53,0.00,0.00}{#1}}
% \renewcommand{\ControlFlowTok}[1]{\textcolor[rgb]{0.00,0.44,0.13}{\textbf{#1}}}
% \renewcommand{\DataTypeTok}[1]{\textcolor[RGB]{144,32,0}{#1}}
% \renewcommand{\DecValTok}[1]{\textcolor[rgb]{0.25,0.63,0.44}{#1}}
% \renewcommand{\DocumentationTok}[1]{\textcolor[rgb]{0.73,0.13,0.13}{\textit{#1}}}
% \renewcommand{\ErrorTok}[1]{\textcolor[rgb]{1.00,0.00,0.00}{\textbf{#1}}}
% \renewcommand{\ExtensionTok}[1]{#1}
% \renewcommand{\FloatTok}[1]{\textcolor[rgb]{0.25,0.63,0.44}{#1}}
% \renewcommand{\FunctionTok}[1]{\textcolor[rgb]{0.02,0.16,0.49}{#1}}
% \renewcommand{\ImportTok}[1]{#1}
% \renewcommand{\InformationTok}[1]{\textcolor[rgb]{0.38,0.63,0.69}{\textbf{\textit{#1}}}}
% \renewcommand{\KeywordTok}[1]{\textcolor[RGB]{0,112,32}{\textbf{#1}}}
% \renewcommand{\NormalTok}[1]{\textcolor[RGB]{65,131,196}{#1}}
% \renewcommand{\OperatorTok}[1]{\textcolor[RGB]{200,0,0}{\textbf{#1}}}
% \renewcommand{\OtherTok}[1]{\textcolor[RGB]{30, 150, 3}{\textbf{#1}}}
% \renewcommand{\PreprocessorTok}[1]{\textcolor[rgb]{0.74,0.48,0.00}{#1}}
% \renewcommand{\RegionMarkerTok}[1]{#1}
% \renewcommand{\SpecialCharTok}[1]{\textcolor[rgb]{0.25,0.44,0.63}{#1}}
% \renewcommand{\SpecialStringTok}[1]{\textcolor[rgb]{0.73,0.40,0.53}{#1}}
% \renewcommand{\StringTok}[1]{\textcolor[RGB]{8, 68, 128}{#1}}
% \renewcommand{\VariableTok}[1]{\textcolor[rgb]{0.10,0.09,0.49}{#1}}
% \renewcommand{\VerbatimStringTok}[1]{\textcolor[rgb]{0.25,0.44,0.63}{#1}}
% \renewcommand{\WarningTok}[1]{\textcolor[rgb]{0.38,0.63,0.69}{\textbf{\textit{#1}}}}
% }
% \makeatother
