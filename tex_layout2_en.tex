\usepackage[grey,utopia]{quotchap}
\usepackage[left=1.25in,right=1in,top=1.2in,bottom=1.2in]{geometry}
\usepackage{fancyhdr}
\usepackage{float}
\usepackage{calc}
\usepackage{lastpage}
\usepackage{amsmath}
\usepackage{amssymb}
\usepackage{hhflxbox}
\usepackage{graphicx}
\usepackage{marvosym}
\usepackage{eclbkbox}
\usepackage[colorlinks=true,linkcolor=blue,citecolor=blue,pagebackref]{hyperref}
%\usepackage{nohyperref}
%\usepackage[french]{babel}
\usepackage[T1]{fontenc}
\usepackage{lmodern} %\Rajouter pour enlever la pixelisation du texte
\usepackage{natbib}
\usepackage{url}
\usepackage[labelsep=endash,textfont=sl]{caption}
\ifx\flagpdf\undefined\graphicspath{{Figures/eps/},{Photos/thumbs/}}%
    \else\ifx\flaglowres\undefined\graphicspath{{Figures/pdf/},{Photos/hires/}}%
        \else\graphicspath{{Figures/pdf/},{Photos/lowres/}}\fi\fi

\usepackage{pdfpages}

%% Mise en page avec espace entre les paragraphes et pas d'indentation
%\setlength{\parskip}{11pt}\setlength{\parindent}{0pt}

\hypersetup{%
    pdftitle={Manual for building tree volume and biomass allometric
equations: from field measurement to prediction},%
    pdfauthor={Nicolas Picard, Laurent Saint-Andr�, Matieu Henry},%
    pdfsubject={copyright FAO et CIRAD},%
    pdfkeywords={allometric equation,biomass table,volume table}}
\fancyhead{}%
\fancyhead[LE]{\thepage}%
\fancyhead[RE]{\leftmark}%
\fancyhead[LO]{\rightmark}%
\fancyhead[RO]{\thepage}%
\fancyfoot{}%
\pagestyle{fancy}

%% pour g�rer les couleurs conditionnelles
\def\ifcouleur#1{\ifx\flagcouleur\undefined\relax\else #1\fi}
\def\ifelsecoul#1#2{\ifx\flagcouleur\undefined #2\else #1\fi}
\def\ifrouge{\ifcouleur{\color{red}}}
\def\ifbleu{\ifcouleur{\color{blue}}}
\def\ifincludegraphics#1#2{\ifx\flagcouleur\undefined\includegraphics[#1]{#2-NB}%
    \else\includegraphics[#1]{#2}\fi}

%% DEBUT DES MODIFICATIONS DE TYPE PACKAGE
\makeatletter
%%% ----------debut de bigcenter.sty--------------
\newskip\@bigflushglue \@bigflushglue = -100pt plus 1fil
\def\bigcenter{\trivlist \bigcentering\item\relax}
\def\bigcentering{\let\\\@centercr\rightskip\@bigflushglue%
\leftskip\@bigflushglue
\parindent\z@\parfillskip\z@skip}
\def\endbigcenter{\endtrivlist}
%%% ----------fin de bigcenter.sty--------------

%% Commande tabcaption
\newcommand{\tabcaption}{\def\@captype{table}\caption}

%% Supprime en-tete et bas de page sur la derniere page d'un chapitre si elle est blanche
\def\cleardoublepage{\clearpage\if@twoside \ifodd\c@page\else
    \hbox{ }
    \thispagestyle{empty}
    \newpage
    \if@twocolumn\hbox{}\newpage\fi\fi\fi}

%% Redefinit le style des en-tetes en petites capitales
\renewcommand{\chaptermark}[1]{
%        \markboth{\sc\small\chaptername\ \thechapter.\ #1}{}}%% sans fancyhdr
        \markboth{\sc\chaptername\ \thechapter.\ #1}{}}%% avec fancyhdr
\renewcommand{\sectionmark}[1]{
%        \markright{\sc\small\thesection\ #1}}%% sans fancyhdr
        \markright{\sc\thesection\ #1}}%% avec fancyhdr

%% Pour mettre toutes les titres de section en police sans s�rif (modification de la classe 'book')
\renewcommand\section{\@startsection {section}{1}{\z@}%
                                   {-3.5ex \@plus -1ex \@minus -.2ex}%
                                   {2.3ex \@plus.2ex}%
                                   {\sffamily\Large\bfseries}}
\renewcommand\subsection{\@startsection{subsection}{2}{\z@}%
                                   {-3.25ex\@plus -1ex \@minus -.2ex}%
                                   {1.5ex \@plus .2ex}%
                                   {\sffamily\large\bfseries}}
\renewcommand\subsubsection{\@startsection{subsubsection}{3}{\z@}%
                                   {-3.25ex\@plus -1ex \@minus -.2ex}%
                                   {1.5ex \@plus .2ex}%
                                   {\sffamily\normalsize\bfseries}}
\renewcommand\paragraph{\@startsection{paragraph}{4}{\z@}%
                                   {3.25ex \@plus1ex \@minus.2ex}%
                                   {-1em}%
                                   {\sffamily\normalsize\bfseries}}
\renewcommand\subparagraph{\@startsection{subparagraph}{5}{\parindent}%
                                   {3.25ex \@plus1ex \@minus .2ex}%
                                   {-1em}%
                                   {\sffamily\normalsize\bfseries}}

%% Red�finit les commandes \tableofcontents, \listoffigures et \listoftables de sorte que
%% - les en-tetes sont en petites capitales
%% - il n'y a pas d'en-tete sur la premiere page
%% (modification de la classe 'book.cls')
\def\chapterannex#1{\chapter*{#1
%        \@mkboth{\sc\small #1}{\sc\small #1}%% sans fancyhdr
        \@mkboth{\sc #1}{\sc #1}%% avec fancyhdr
        \addcontentsline{toc}{chapter}{\numberline {}#1}}%% ligne modifi�e
    \thispagestyle{plain}}%% ligne ajout�e
\renewcommand\tableofcontents{%
    \if@twocolumn\@restonecoltrue\onecolumn\else\@restonecolfalse\fi
    \chapterannex{\contentsname}
    \@starttoc{toc}%
    \if@restonecol\twocolumn\fi}
\renewcommand\listoffigures{%
    \if@twocolumn\@restonecoltrue\onecolumn\else\@restonecolfalse\fi
    \chapterannex{\listfigurename}
    \@starttoc{lof}%
    \if@restonecol\twocolumn\fi}
\renewcommand\listoftables{%
    \if@twocolumn\@restonecoltrue\onecolumn\else\@restonecolfalse\fi
    \chapterannex{\listtablename}
    \@starttoc{lot}%
    \if@restonecol\twocolumn\fi}

%% Red�finit l'environnement \thebibliography de sorte que
%% - les en-tetes sont en petites capitales
%% - il n'y a pas d'en-tete sur la premiere page
%% (modification du package 'natbib.sty', qui est d�j� une modification de la classe 'book.cls')
\renewcommand\bibsection{%
    \if@twocolumn\@restonecoltrue\onecolumn\else\@restonecolfalse\fi
    \chapterannex{\bibname}
    \if@restonecol\twocolumn\fi}
%% red�finit les espaces entre r�f�rences
\setlength{\bibhang}{0cm}\setlength{\bibsep}{11pt}
%% astuce pour r�cup�rer le nombre total de r�f�rences bibliographiques
\newread\fichier
\openin\fichier=\jobname.nbr \ifeof\fichier
    \typeout{Picolas's Warning: Fichier \jobname.nbr non trouve}
    \def\numberofrefs{255 }
    \else \read\fichier to \numberofrefs \closein\fichier\relax%
    \fi

%% Red�finit la commande \float@listhead qui est l'�quivalent de \listoffigures
%% pour le package 'float' (modification du package 'float')
\renewcommand\float@listhead[1]{\chapterannex{#1}}%

%%% ----------environnement fil rouge---------------
%\def\ext@filrouge{lor}
\newcommand\listfilrougename{List of Red lines}
\newcommand\listoffilrouge{%
    \if@twocolumn\@restonecoltrue\onecolumn\else\@restonecolfalse\fi
    \chapterannex{\listfilrougename}%
    \@starttoc{lor}%
    \if@restonecol\twocolumn\fi}
\let\l@filrouge\l@figure
\def\logofilrouge{\vrule height 0.4mm depth 0mm width 1cm\hskip-0.1mm{\Large\Denarius}\!%
    \raisebox{-0.1cm}{\vrule height 0.4mm depth 0mm width 12.3cm}}%
\newcounter{filrouge}
\def\mycontentsline#1#2#3{\contentsline{#1}{#2}{#3}{}} %% � cause de hyperref qui red�finit cette commande avec 3 arguments
\newenvironment{filrouge}[2]{%
    \refstepcounter{filrouge}%
    \addtocontents{lor}{\protect\mycontentsline {filrouge}%
        {\string\hyperlink{#2}{\string\numberline\space{\thefilrouge}{\string\ignorespaces\space #1}}}{\thepage}}%
    \par\vskip12pt\noindent{\ifrouge%
        \hskip-2ex\raisebox{-0.6ex}{\bellybox:{\arabic{filrouge}}}\logofilrouge}%
        \nopagebreak\par\smallskip\noindent\textbf{#1}\nopagebreak\par\bigskip\hypertarget{#2}{\label{#2}}}%
    {\nopagebreak\par\smallskip\noindent{\ifrouge\logofilrouge}\par\vskip12pt}
%%% ----------fin de l'environnment fil rouge-----

%% FIN DES MODIFICATIONS DE TYPE PACKAGE
\makeatother%

%% d�finition du flottant 'photo'
\newfloat{photo}{tbph}{pho}
\floatname{photo}{\textsc{Photo}}
\renewcommand{\thephoto}{\thechapter.\arabic{photo}}
%% routines pour la mise en page des photos
\newlength{\marginphoto}\setlength{\marginphoto}{1mm} %% marge s�parant les photos
\setlength{\fboxsep}{0pt}\setlength{\fboxrule}{0.3pt} %% �paisseur du cadre noir entourant les photos
\newlength{\largeurphoto}
\newlength{\largeurdispophotos}
\newcommand{\includeonephoto}[1]{%
    \setlength{\largeurphoto}{\textwidth-2\fboxrule}
    \framebox{\includegraphics[width=\largeurphoto]{#1}}}
\newcommand{\includetwophotos}[4]{%
    \setlength{\largeurdispophotos}{\textwidth-\fboxrule*\real{4}-\marginphoto}%
    \setlength{\largeurphoto}{\largeurdispophotos*\real{#2}}%
    \framebox{\includegraphics[width=\largeurphoto]{#1}}%
    \hskip\marginphoto%
    \setlength{\largeurphoto}{\largeurdispophotos*\real{#4}}%
    \framebox{\includegraphics[width=\largeurphoto]{#3}}}
\newcommand{\includethreephotos}[6]{%
    \setlength{\largeurdispophotos}{\textwidth-\fboxrule*\real{6}-\marginphoto*\real{2}}%
    \setlength{\largeurphoto}{\largeurdispophotos*\real{#2}}%
    \framebox{\includegraphics[width=\largeurphoto]{#1}}%
    \hskip\marginphoto%
    \setlength{\largeurphoto}{\largeurdispophotos*\real{#4}}%
    \framebox{\includegraphics[width=\largeurphoto]{#3}}%
    \hskip\marginphoto%
    \setlength{\largeurphoto}{\largeurdispophotos*\real{#6}}%
    \framebox{\includegraphics[width=\largeurphoto]{#5}}}

%% routines simples
\bibpunct{(}{)}{;}{a}{,}{,}
%\newenvironment{encadre}{%
%    \par\vskip12pt\begin{breakbox}\noindent{\LARGE\Pointinghand\ }%
%    \begin{minipage}[t]{12cm}\raggedright}{%
%    \end{minipage}\end{breakbox}\par\vskip12pt}
\def\vect#1{\mathbf{#1}}
\def\matx#1{\mathbf{#1}}
\def\tr#1{{}^{\mathrm{t}}#1}
\def\numeros{n\up{os}\kern.2em}
\def\d{\mathsf{d}}
\definecolor{orange}{rgb}{0.94,0.64,0.10}
\newcommand{\decimal}[2]{\mbox{#1.#2}} %% pour �viter qu'il y ait un espace apr�s la virgule dans les nombres d�cimaux
\newcommand{\gloss}[2]{\paragraph{#1.} #2}
\newcommand{\R}[1]{\par\medskip\noindent{\ifrouge\small\tt #1}\par\medskip\noindent}%
\newcommand{\Rout}[1]{\par\medskip\noindent{\ifbleu\small\tt #1}\par\medskip\noindent}
%\NoAutoSpaceBeforeFDP
\def\bidouille#1{#1}
